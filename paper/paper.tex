\documentclass[fleqn,10pt]{wlscirep}
\title{Scientific Reports Title to see here (20 words or less)}

\author[1,*]{Yair Daon}
\author[2]{Hao Ye}
\author[3]{George Sugihara}
\affil[1]{NYU, Courant Institute of Mathematical Sciences, New York, NY 10012 USA}
\affil[2]{University of California San Diego, Scripps Institution of Oceanography, La Jolla, CA, postcode, country}

\affil[*]{yair.daon@gmail.com}

\affil[+]{these authors contributed equally to this work}

%\keywords{Keyword1, Keyword2, Keyword3}

\begin{abstract}
  We explore the change in prediction skill made by weighting 
\end{abstract}
\begin{document}

\flushbottom
\maketitle
% * <john.hammersley@gmail.com> 2015-02-09T12:07:31.197Z:
%
%  Click the title above to edit the author information and abstract
%
\thispagestyle{empty}

\noindent Please note: Abbreviations should be introduced at the first
mention in the main text – no abbreviations lists. Suggested structure
of main text (not enforced) is provided below.

\section*{Introduction}
Nonlinear prediciton using S-maps and simplex projection are by now an
establised set of techniques used for analyzing natural systems.


We introduce a measure of prediction uncertainty for the S-map. We are
given one or several time series which we use to reconstruct an $E$
dimensional attractor. Denote the state on the attractor
$\mathbf{x}^{(t)} = (x_1^{(t)},...,x_E^{(t)})$ with $t=1,...,T$. In
the univariate case we take time lags of the same time series, so
$\mathbf{(t)} = (x^{(t)},...,x^{(t-E+1)})$. For every $t$, we are also
given an observation $y^{(t)}$. Note that in many cases, we may want
to predict the time evolution of (say) the first coordinate and then
$y^{(t)} := x_1^{(t-1)}$. The S-map prediction for a new $\mathbf{x}$
with a given nonlinearity parameter $\theta$ is found by first solving
the (local) least squares problem for $\beta$:
\begin{equation}
  \beta := \arg \min_{\hat{\beta}} \|W(X \hat{\beta} - \mathbf{y}) \|_2^2,
\end{equation}
with
\begin{equation*}
  X =
  \begin{bmatrix}
    -& -& -& \mathbf{x}^{(1)} & - & - & - \\
     &  &  &  \vdots         &   &   &   \\
    -& -& -& \mathbf{x}^{(T)} & - & - & - \\    
  \end{bmatrix},
  \mathbf{y} =
  \begin{bmatrix}
    y^{(1)} \\
    \vdots \\
    y^{(T)}
  \end{bmatrix}
  \text{ and } W := diag( w_1,...,w_T), w_t: =\exp( -\theta \| \mathbf{x} - \mathbf{x}^{(t)}\| )
\end{equation*}
Given $\beta$, we predict $y = \beta' \mathbf{x} = \sum_{i=1}^E
\beta_i x_i$. The novel uncertainty measure we intruduce is simply
defined as
\begin{equation}\label{eq:uncertainty}
  \text{Uncertainty}(\mathbf{x}) := \sum_{t=1}^T w_t (y - y^{(t)})^2
  \bigg / \sum_{t=1}^T w_t = \sum_{t=1}^T w_t (\beta' \mathbf{x} -
  y^{(t)})^2 \bigg / \sum_{t=1}^T w_t.
\end{equation}
This is in contrast to the similar \emph{fitting error} that we found
insufficient
\begin{equation*}
\text{Fitting Error}(\mathbf{x}) := \sum_{t=1}^T w_t ( \beta'
\mathbf{x}^{(t)} - y^{(t)})^2 \bigg / \sum_{t=1}^T w_t.
\end{equation*}
The former represents the divergence in trajectories of the time
series, whereas the latter merely represents a fitting error, hence
underestimating the prediction uncertainty associated with
$\mathbf{x}$.
%% The Introduction section, of referenced
%% text\cite{Figueredo:2009dg} expands on the background of the work
%% (some overlap with the Abstract is acceptable). The introduction
%% should not include subheadings.

\section*{Results}
Using the uncertainty measure described in \eqref{eq:uncertainty} we
approach the problem of predicting Chlorophyll abundance (denoted Chl)
in the waters of southern California. For a thorough introduction, see
\cite{that paper}. It was found that six variables are causally
related to the dynamics of Chl. We consider models that contain the
Chl data at the time of observation, denoted Chl(t) and three other
variables that may or may not be lagged. We take lags of one or two
weeks so we have $6 \cdot 3 + 2 = 20$ variables to choose --- lags of
0,1, or 2 weeks for the environmental variables plus lag of 1 or 2
weeks Chl. Since the embedding dimension was found to be $E=4$, we
have $\binom{20}{3} = 1140$ possible models to choose from. For each
prediction time, we can choose a fraction of the least uncertain (most
confident) models to predict and average these predictions.


\subsection*{Subsection}

Example text under a subsection. Bulleted lists may be used where
appropriate, e.g.

\begin{itemize}
\item First item
\item Second item
\end{itemize}

\subsubsection*{Third-level section}
 
Topical subheadings are allowed.

\section*{Discussion}

The Discussion should be succinct and must not contain subheadings.

\section*{Methods}

Topical subheadings are allowed. Authors must ensure that their
Methods section includes adequate experimental and characterization
data necessary for others in the field to reproduce their work.

\bibliography{sample}

\noindent LaTeX formats citations and references automatically using
the bibliography records in your .bib file, which you can edit via the
project menu. Use the cite command for an inline citation, e.g.
\cite{Figueredo:2009dg}.

\section*{Acknowledgements (not compulsory)}

Acknowledgements should be brief, and should not include thanks to
anonymous referees and editors, or effusive comments. Grant or
contribution numbers may be acknowledged.

\section*{Author contributions statement}

Must include all authors, identified by initials, for example:
A.A. conceived the experiment(s), A.A. and B.A. conducted the
experiment(s), C.A. and D.A. analysed the results.  All authors
reviewed the manuscript.

\section*{Additional information}

To include, in this order: \textbf{Accession codes} (where
applicable); \textbf{Competing financial interests} (mandatory
statement).

The corresponding author is responsible for submitting a
\href{http://www.nature.com/srep/policies/index.html#competing}{competing
  financial interests statement} on behalf of all authors of the
paper. This statement must be included in the submitted article file.

\begin{figure}[ht]
\centering
\includegraphics[width=\linewidth]{stream}
\caption{Legend (350 words max). Example legend text.}
\label{fig:stream}
\end{figure}

\begin{table}[ht]
\centering
\begin{tabular}{|l|l|l|}
\hline
Condition & n & p \\
\hline
A & 5 & 0.1 \\
\hline
B & 10 & 0.01 \\
\hline
\end{tabular}
\caption{\label{tab:example}Legend (350 words max). Example legend text.}
\end{table}

Figures and tables can be referenced in LaTeX using the ref command,
e.g. Figure \ref{fig:stream} and Table \ref{tab:example}.

\end{document}
